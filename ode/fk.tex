\documentclass[a4paper, 11pt, notitlepage, english]{article}

\usepackage{babel}
\usepackage[utf8]{inputenc}
\usepackage[T1]{fontenc, url}
\usepackage{textcomp}
\usepackage{amsmath, amssymb}
\usepackage{amsbsy, amsfonts}
\usepackage{graphicx, color, xcolor}
\usepackage{verbatim, listings, fancyvrb}
\usepackage{parskip}
\usepackage{framed}
\usepackage{amsmath}
\usepackage{multicol}
\usepackage{url}
\usepackage{flafter}
\usepackage{simplewick}
\usepackage{amsthm}
\usepackage{bbold}


\usepackage{caption}
\DeclareCaptionLabelSeparator{colon}{. }
\renewcommand{\captionfont}{\small\sffamily}
\renewcommand{\captionlabelfont}{\bf\sffamily}
\usepackage{float}
%\floatstyle{ruled}
%\restylefloat{figure}
\setlength{\captionmargin}{20pt}
%\addto\captionsenglish{\renewcommand{\figurename}{Fig.}}
\usepackage{bigstrut}
\setlength{\tabcolsep}{12pt}


\newtheorem{theorem}[]{Wick's Theorem}[]

\DeclareUnicodeCharacter{00A0}{~}

\definecolor{javared}{rgb}{0.6,0,0} % for strings
\definecolor{javagreen}{rgb}{0.25,0.5,0.35} % comments
\definecolor{javapurple}{rgb}{0.5,0,0.35} % keywords
\definecolor{javadocblue}{rgb}{0.25,0.35,0.75} % javadoc

\lstset{language=python,
basicstyle=\ttfamily\scriptsize,
keywordstyle=\color{javapurple},%\bfseries,
stringstyle=\color{javared},
commentstyle=\color{javagreen},
morecomment=[s][\color{javadocblue}]{/**}{*/},
morekeywords={super, with},
% numbers=left,
% numberstyle=\tiny\color{black},
stepnumber=2,
numbersep=10pt,
tabsize=2,
showspaces=false,
captionpos=b,
showstringspaces=false,
frame= single,
breaklines=true}

\usepackage{geometry}
\geometry{headheight=0.01mm}
\geometry{top=20mm, bottom=20mm, left=34mm, right=34mm}

\renewcommand{\arraystretch}{2}
\setlength{\tabcolsep}{10pt}
\makeatletter
\renewcommand*\env@matrix[1][*\c@MaxMatrixCols c]{%
  \hskip -\arraycolsep
  \let\@ifnextchar\new@ifnextchar
  \array{#1}}
%
% Definering av egne kommandoer og miljøer
%
\newcommand{\dd}[1]{\ \text{d}#1}
\newcommand{\f}[2]{\frac{#1}{#2}} 
\newcommand{\beq}{\begin{equation}}
\newcommand{\eeq}{\end{equation}}
\newcommand{\bra}[1]{\langle #1|}
\newcommand{\ket}[1]{|#1 \rangle}
\newcommand{\braket}[2]{\langle #1 | #2 \rangle}
\newcommand{\brakket}[2]{\langle #1 || #2 \rangle}
\newcommand{\braup}[1]{\langle #1 \left|\uparrow\rangle\right.}
\newcommand{\bradown}[1]{\langle #1 \left|\downarrow\rangle\right.}
\newcommand{\av}[1]{\left| #1 \right|}
\newcommand{\op}[1]{\hat{#1}}
\newcommand{\braopket}[3]{\langle #1 | {#2} | #3 \rangle}
\newcommand{\ketbra}[2]{\ket{#1}\bra{#2}}
\newcommand{\pp}[1]{\frac{\partial}{\partial #1}}
\newcommand{\ppn}[1]{\frac{\partial^2}{\partial #1^2}}
\newcommand{\up}{\left|\uparrow\rangle\right.}
\newcommand{\upup}{\left|\uparrow\uparrow\rangle\right.}
\newcommand{\down}{\left|\downarrow\rangle\right.}
\newcommand{\downdown}{\left|\downarrow\downarrow\rangle\right.}
\newcommand{\updown}{\left|\uparrow\downarrow\rangle\right.}
\newcommand{\downup}{\left|\downarrow\uparrow\rangle\right.}
\newcommand{\bupup}{\left.\langle\uparrow\uparrow\right|}
\newcommand{\bdowndown}{\left.\langle\downarrow\downarrow\right|}
\newcommand{\bupdown}{\left.\langle\uparrow\downarrow\right|}
\newcommand{\bdownup}{\left.\langle\downarrow\uparrow\right|}
\renewcommand{\d}{{\rm d}}
\newcommand{\Res}[2]{{\rm Res}(#1;#2)}
\newcommand{\To}{\quad\Rightarrow\quad}
\newcommand{\eps}{\epsilon}
\newcommand{\inner}[2]{\langle #1 , #2 \rangle}
\renewcommand{\u}{\uparrow}
% \renewcommand{\d}{\downarrow}
\newcommand{\dddd}{\d\d\d\d}
\newcommand{\uddd}{\u\d\d\d}
\newcommand{\dudd}{\d\u\d\d}
\newcommand{\ddud}{\d\d\u\d}
\newcommand{\dddu}{\d\d\d\u}
\newcommand{\uudd}{\u\u\d\d}
\newcommand{\udud}{\u\d\u\d}
\newcommand{\uddu}{\u\d\d\u}
\newcommand{\duud}{\d\u\u\d}
\newcommand{\dudu}{\d\u\d\u}
\newcommand{\dduu}{\d\d\u\u}
\newcommand{\uuud}{\u\u\u\d}
\newcommand{\uudu}{\u\u\d\u}
\newcommand{\uduu}{\u\d\u\u}
\newcommand{\duuu}{\d\u\u\u}
\newcommand{\uuuu}{\u\u\u\u}
\newcommand{\m}{\text{-}}
\newcommand{\ui}{{\u_1}}
\newcommand{\uii}{{\u_2}}
\newcommand{\uiii}{{\u_3}}
\newcommand{\di}{{\d_1}}
\newcommand{\dii}{{\d_2}}
\newcommand{\diii}{{\d_3}}

\newenvironment{psmallmatrix}
  {\left(\begin{smallmatrix}}
  {\end{smallmatrix}\right)}

\newenvironment{bsmallmatrix}
  {\left[\begin{smallmatrix}}
  {\end{smallmatrix}\right]}



\newcommand{\bt}[1]{\boldsymbol{#1}}
\newcommand{\mat}[1]{\textsf{\textbf{#1}}}
\newcommand{\I}{\boldsymbol{\mathcal{I}}}
\newcommand{\p}{\partial}

\title{The 4-variable Fenton Karma Model (FK4V)}

\begin{document}

\maketitle

The 4-variable Fenton Karma is a phenomenological model consisting of 4 states, these are:
\begin{itemize}
    \item The membrane potentital, $V$
    \item The gating variables, $v$, $w$ and $d$
\end{itemize}

And the membrane potentital is driven by three currents
$$I_{\rm fi}, \qquad I_{\rm si}, \qquad I_{\rm so}.$$
Where the subindex denotes the \emph{fast inward}, \emph{slow inward} and \emph{slow outward} currents. 

The fast inward current is responsible for the depolarization of the membrane and only depends on the gating variable, $v$, which is a inactivation-reactivation gate. This gating variable is responsible for inactivation of the fast inward current when the cell depolarizes - and of the reactivation of the $I_{\rm fi}$ when the cell repolarizes. 

The slow outward current is responsible for the repolarization of the membrane, it is mainly drivven by a time-independant potassium current.

Finally, the slow inward current mainly consists of calcium current, and mainly balances $I_{\rm so}$, causing the plateau of the action potential. It depends on the gating variable $w$, which is responsible for its inactivation and reactivation.

It is important to emphasize that while these three currents correspond to physical currents, the model is mainly phenomenological, and so any interpretion should be seen as mainly illustrative---the $I_{\rm fi}$ is \emph{not} really a sodium current.

\section*{Equations}

We specify the membrane potential in a dimensionless (scaled) form, so that 
$$u \equiv \frac{V-V_0}{V_{\rm fi}-V_0}.$$
Where $V_0$ is the resting membrane potential and $V_{\rm fi}$ is the Nernst-potential of the fast-inward current, which will correspond to the peak of the membrane potential. With this definition, we have that the membrane potential varies as $u \in [0,1]$.

The membrane potential is driven by these three currents, as well as an applied stimulus current, so we have
$$\frac{\d u }{\d t} = -I_{\rm fi} - I_{\rm so} - I_{\rm si}.$$

\section*{Expressions for the currents}

The fast inward current is given by
$$J_{\rm fi} = -\frac{v}{\tau_d}H(u-u_c)(1-u)(u-u_c).$$






\end{document}
